\documentclass[12pt, a4paper]{article}
\addtolength{\oddsidemargin}{-.875in}
\addtolength{\evensidemargin}{-.875in}
\addtolength{\textwidth}{1.75in}
\addtolength{\topmargin}{-.875in}
\addtolength{\textheight}{1.75in}

\usepackage{indentfirst}
\usepackage{graphicx}
\usepackage{amsmath}


\begin{document}
\noindent
Nicholas Garrett\\ \\
Professor Zhu\\ \\
MATH 4441\\ \\
10/27/2021\\ \\


\begin{center}
	\centering{	Homework 8\\ }
\end{center}

\noindent
1.\\
See code\\

The error norm for Gauss-Seidel after a single iteration was 31.82, and the error norm for Jacobi (after 2 iterations) was 6.624, so Jacobi seems to be converging more slowly after two iterations than Gauss-Seidel after a single iteration. \\ \\ \\

2.\\

For the matrix to be positive, \(a > 0\).  Then, because the convergence of the Jacobi iteration sequence requires that \(\rho(A) < 1\).  The eigenvalues of A are : \(1-a\) and \(2a+1\). For the absolute value of the maximum eigenvalue to be \(< 1\), a cannot be any value, as the maximum eigenvalue is \(> 1\) for every value of a.  Therefore, a can be any value greater than zero.  \\ \\ \\

3.a.\\
 \(M = \alpha^{-1}\) \\
B is the iteration matrix,  and \(B = I-\alpha A\)\\ \\ \\


3.b.i.\\
For the iteration matrix to converge, \(\rho(B) < 1\).  B = I - \(M^{-1}A\), and by part A, we find that B = I - \(\alpha\)A.  For \(\rho(B)\) to be less than 1, \( max|\lambda | = |I-\lambda (I-\alpha A)|<1 \),   which can be guaranteed if \(\alpha = \lambda\). \\ \\ \\


3.b.ii.\\
By theorem 3, as A is symmetric positive definite, ( I use f because I cannot find the greek letter used in the notes) \( g(\alpha) = f(x_k + \alpha p_k) = \frac{1}{2}(x_k + \alpha p_k)^T A(x_k + \alpha p_k)-(x_k + \alpha p_k)^Tb\), which is what we want to minimize for \( \alpha\).\\ 

By substituting alpha into the equations, we get:\\ \(  g(\frac{2}{\lambda _ 1 + \lambda _ n}) = f(x_k + \alpha p_k) = \frac{1}{2}(x_k + \frac{2}{\lambda _ 1 + \lambda _ n} p_k)^T A(x_k + \frac{2}{\lambda _ 1 + \lambda _ n} p_k)-(x_k + \frac{2}{\lambda _ 1 + \lambda _ n} p_k)^Tb\)

   ...

\end{document}  