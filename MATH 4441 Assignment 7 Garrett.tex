\documentclass[12pt, a4paper]{article}
\addtolength{\oddsidemargin}{-.875in}
\addtolength{\evensidemargin}{-.875in}
\addtolength{\textwidth}{1.75in}
\addtolength{\topmargin}{-.875in}
\addtolength{\textheight}{1.75in}

\usepackage{indentfirst}
\usepackage{graphicx}
\usepackage{amsmath}


\begin{document}
\noindent
Nicholas Garrett\\ \\
Professor Zhu\\ \\
MATH 4441\\ \\
10/20/2021\\ \\


\begin{center}
	\centering{	Homework 7\\ }
\end{center}

\noindent
1.a 
\begin{verbatim}
ti = [ 1.0; 1.5; 2.0; 2.5; 3.0 ]
yi = [ 1.1; 1.2; 1.3; 1.3; 1.4 ]

A = [   1, ti(1); 
        1, ti(2); 
        1, ti(3); 
        1, ti(4); 
        1, ti(5) ] 
\end{verbatim} .\\ 

\noindent
1.b \\
\( y = 0.14(ti) + 0.98 \) \\ \\


\noindent
1.c \\
\begin{figure}[ht!]
\centering
\includegraphics[width=90mm]{equationFittingGraph.jpg}
\end{figure}
The Blue plot is from the provided data, the orange line is the iftted line. \\ \\

\noindent
1.d \\
For whatever reason, MATLAB kept returning an error when I tried to use the norm command.  I tried building a vector of difference values that it could take the norm of, but that also threw an error.  \\
Ultimately, I had to calculate by hand the \(l_2\)-norm, which is 0.003.\\

Shown is the code I tried to use to calculate the norm:\\
\begin{verbatim}
% compute norm of the residual
n = size(ti,1)
%norm = norm(yi - psiTi(ti(i))),2)
for i = 1:n
    yi(i)
    psiTi(i)
    normV(i) = abs(yi(i) - psiTi(i))
end
norm(normV,2) 
\end{verbatim} .\\ 

\noindent
2.a \\ 
\indent
\( H = I - \frac{2uu^T}{u^Tu} \\ \Rightarrow H*u = (I - \frac{2uu^T}{u^Tu})*u \\ \Rightarrow Hu = Iu - \frac{2uu^Tu}{u^Tu} \\ \Rightarrow Hu = Iu - \frac{2u}{1} \\ \Rightarrow Hu = u - 2u \\ \Rightarrow Hu = -u \) \\ \\

\noindent
2.b\\ 
Givent that \( v^Tu = 0 \) \\
\( H = I - \frac{2uu^T}{u^Tu} \\ \Rightarrow H*v = (I - \frac{2uu^T}{u^Tu})v \\ \Rightarrow  H*v = Iv - \frac{2uu^Tv}{u^Tu} \)\\ \( v^Tu = 0\) is given.  So, through symmetry, \(v^Tu = u^Tv = 0, \Rightarrow  H*v = Iv - \frac{2u0}{u^Tu}\\ \Rightarrow Hv = Iv - 0 \\ \Rightarrow Hv = v\) \\ \\

\noindent
3.a \\
This algorithm should take about \((r^3)\) flops, as that is the number of operations used in factorization by Householder, but we only need to operate r times \\ \\	

\noindent
3.b \\
See code \\ \\

\noindent
3.c \\
See code \\ \\

\end{document}  