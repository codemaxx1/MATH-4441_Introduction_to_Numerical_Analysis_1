\documentclass[12pt, a4paper]{article}
\addtolength{\oddsidemargin}{-.875in}
\addtolength{\evensidemargin}{-.875in}
\addtolength{\textwidth}{1.75in}
\addtolength{\topmargin}{-.875in}
\addtolength{\textheight}{1.75in}

\usepackage{xcolor}
\usepackage{indentfirst}
\usepackage{graphicx}


\begin{document}

\noindent
Nick Garrett\\ \\
Professor Zhu\\ \\
Math 4441\\ \\
9/8/2021\\ \\


\begin{center}
	\centering{Project 2\\ }
\end{center}


1.\\
	A better formula would be \( f'(x_{0}) = \frac{2}{h}sin(\frac{(x_{0}+h) - x_{0}}{2}) cos(\frac{(x_{0}+ h)+x_{0}}{2}) \)
	Which, when compared to the exact derrivative, follows very closely to what would be expected.  
	\begin{center}
		\centering
		\includegraphics[width=100mm]{math4441_homework2_Nicholas_Garrett_problem1_error}
	\end{center}
	The difference in accuracy could be explained by seeing how the original example subtracts one term from the other.  This caused every other order of the Taylor approximation to cancel.  However, by using the triganometric identity to remove the subraction, that difference error is effectively removed.\\ \\
	
2.\\

a.\\

	\( \sqrt{x+ \frac{1}{x}} - \sqrt{x -  \frac{1}{x}} \) where \(x >> 1\) would lead to two numbers close to equal being subtracted from one another, leading to a difference error. \\
	A better way to write this would be to multiply \( \sqrt{x - \frac{1}{x}} \) by \( \frac{\sqrt{x + \frac{1}{x}}}{\sqrt{x + \frac{1}{x}}} \), which would make \( \frac{\sqrt{x^2 - 1}}{2} \) \\
	So,  \( \sqrt{x+ \frac{1}{x}} - \sqrt{x -  \frac{1}{x}} \Longrightarrow  \sqrt{x+ \frac{1}{x}} - \frac{\sqrt{x^2 - 1}}{\sqrt{2}} \) where \( x>>1 \) \\

b.\\
	The problem with this... problem is that a very small number is in the denominator.  As the denominator approaches zero, the value grows.  \\
	\( \sqrt{\frac{1}{a^2} + \frac{1}{b^2}} = \sqrt{\frac{a^2 + b^2}{a^2b^2}} = \frac{\sqrt{a^2 + b^2}}{ab}\)  \\
	So, a better way to write this formula would be \( \frac{\sqrt{a^2 + b^2}}{ab} \) where a \( \approx \) 0 and b \( \approx \) 1. \\


c.\\
	Because of the subtraction, a cancellation error would occur.
	A better way to write it would be \( ln(\sqrt{2x(1 - 2\sqrt{x^2 -1}) -1 } ) \)\\  \\ \\
	

3.\\

a.\\ 
	If \( a \approx b \), this will make the equation simplify to \( x+y=1 \) and \( y+x = 0 \), which is impossible.\\
\\

b.\\  
 From the matrix, we get  \( ax+by = 1 \) and \( bx + ay = 0 \) which can be rewritten to state \( x=\frac{a}{a^2 - b^2}  \) and \( y=\frac{b}{b^2 - a^2} \).  As a and b are presumed to be provided, we can solve for x and y.  This would then lead us to finding z.\\
 So, \( z =  \frac{a}{(a-b)(a+b)} +  \frac{b}{(b-a)(a+b)} \)
 



\end{document}  