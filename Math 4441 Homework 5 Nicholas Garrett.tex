\documentclass[12pt, a4paper]{article}
\addtolength{\oddsidemargin}{-.875in}
\addtolength{\evensidemargin}{-.875in}
\addtolength{\textwidth}{1.75in}
\addtolength{\topmargin}{-.875in}
\addtolength{\textheight}{1.75in}

\usepackage{indentfirst}
\usepackage{graphicx}
\usepackage{xcolor}
\usepackage{amsmath} 

\begin{document}
\noindent
Nicholas Garrett\\ \\
Professor Zhu\\ \\
Math 4441\\ \\
9/29/2021\\ \\


\begin{center}
	\centering{	Assignment 5\\ }
\end{center}

1.\\

\(
\begin{bmatrix}
x_1 & -x_2 & 3x_3 & 2\\
x_1 & x_2 & 0 & 4 \\
3x_1 & -2x^2 & x_3 &1
\end{bmatrix}
\)
\(\Rightarrow\)
\(
\begin{bmatrix}
x_1 & -x_2 & 3x_3 & 2\\
0 & x_2 & -3x_3 & 2 \\
0 & 0 & 13x_3 & 12
\end{bmatrix}
\)
Which gives\\

 \( x_1 = 21/13\), \( x_2 = 31/13\), and \( x_1 = 12/13\) \\ \\ \\

2.\\

Given the information, we can re-write the equation to be \( LUx = Tb \Rightarrow x= L^{-1}U^{-1} * T *b\)\\

Because \(TA = LU,  \,\,L^{-1}U^{-1} = T^{-1}A^{-1} \\
\Rightarrow x= L^{-1}U^{-1} * T *b  \\
\Rightarrow x =L^{-1}U^{-1}*L_TU_T*b  \) \\ \\ \\

3.\\
\begin{verbatim}
function [] = crdout(A)
        
         %size of matrix
         n = length(A);
        
        %put first row of A into lower matrix
        L(:, 1) = A(:, 1);
        
        %put pivots of 1 into Upper matrix
        for i = 1:n
            U(i, i) = 1;
        end
        
        %scale upper matrix
        U(1,:) = A(1, :) / L(1, 1);

        %iterate through rows
        for i = 2:n
        	%
             for j = 2:i
                  %assemble lower matrix
                  L(i, j) = A(i, j) - L(i, 1:1:j - 1) * U(1:1:j - 1, j);
            end
            
            for j = i + 1:n
            	   %assemble upper matrix 
                    U(i, j) = (A(i, j) - L(i, 1:1:i - 1) * U(1:1:i - 1, j)) / L(i, i);
            end
        end
        
        L
        U
        
end
\end{verbatim}

\end{document}  